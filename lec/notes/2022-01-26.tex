\documentclass[12pt, leqno]{article}
\usepackage{fancyhdr}
\usepackage[letterpaper=true,colorlinks=true,linkcolor=black]{hyperref}

\usepackage{amsfonts}
\usepackage{amsmath}
\usepackage{amssymb}
\usepackage{color}
\usepackage{tikz}
\usepackage{pgfplots}
\usepackage{listings}
%\usepackage{courier}
%\usepackage[utf8]{inputenc}
%\usepackage[russian]{babel}

\lstdefinelanguage{Julia}%
  {morekeywords={abstract,break,case,catch,const,continue,do,else,elseif,%
      end,export,false,for,function,immutable,import,importall,if,in,%
      macro,module,otherwise,quote,return,switch,true,try,type,typealias,%
      using,while},%
   sensitive=true,%
   alsoother={$},%
   morecomment=[l]\#,%
   morecomment=[n]{\#=}{=\#},%
   morestring=[s]{"}{"},%
   morestring=[m]{'}{'},%
}[keywords,comments,strings]%

\lstset{
  numbers=left,
  basicstyle=\ttfamily\footnotesize,
  numberstyle=\tiny\color{gray},
  stepnumber=1,
  numbersep=10pt,
}

\newcommand{\iu}{\ensuremath{\mathrm{i}}}
\newcommand{\bbR}{\mathbb{R}}
\newcommand{\bbC}{\mathbb{C}}
\newcommand{\calV}{\mathcal{V}}
\newcommand{\calE}{\mathcal{E}}
\newcommand{\calG}{\mathcal{G}}
\newcommand{\calW}{\mathcal{W}}
\newcommand{\calP}{\mathcal{P}}
\newcommand{\macheps}{\epsilon_{\mathrm{mach}}}
\newcommand{\matlab}{\textsc{Matlab}}
\newcommand{\uQ}{\underline{Q}}
\newcommand{\uR}{\underline{R}}

\newcommand{\ddiag}{\operatorname{diag}}
\newcommand{\fl}{\operatorname{fl}}
\newcommand{\nnz}{\operatorname{nnz}}
\newcommand{\tr}{\operatorname{tr}}
\renewcommand{\vec}{\operatorname{vec}}

\newcommand{\vertiii}[1]{{\left\vert\kern-0.25ex\left\vert\kern-0.25ex\left\vert #1
    \right\vert\kern-0.25ex\right\vert\kern-0.25ex\right\vert}}
\newcommand{\ip}[2]{\langle #1, #2 \rangle}
\newcommand{\ipx}[2]{\left\langle #1, #2 \right\rangle}
\newcommand{\order}[1]{O( #1 )}

\newcommand{\kron}{\otimes}


\newcommand{\hdr}[1]{
  \pagestyle{fancy}
  \lhead{Bindel, Spring 2022}
  \rhead{Numerical Analysis}
  \fancyfoot{}
  \begin{center}
    {\large{\bf #1}}
  \end{center}
  \lstset{language=Julia,columns=flexible}  
}


\begin{document}
\hdr{2020-01-26}

\section{Matrices}

From your linear algebra background, you should know a matrix as a
representation of a linear map.  A matrix can {\em also} represent a
bilinear function mapping two vectors into the real numbers (or
complex numbers for complex vector spaces):
\[
  (v, w) \rightarrow w^* A v.
\]
Symmetric matrices also represent {\em quadratic forms} mapping
vectors to real numbers
\[
  \phi(v) = v^* A v.
\]
We say a symmetric matrix $A$ is {\em positive definite} if the
corresponding quadratic form is positive definite, i.e.
\[
  v^* A v \geq 0 \mbox{ with equality iff } v = 0.
\]
We will talk more about matrices as representations of linear maps,
bilinear forms, and quadratic forms in the next lecture.

Many ``rookie mistakes'' in linear algebra involve forgetting ways in
which matrices differ from scalars:
\begin{itemize}
\item Not all matrices are square.
\item Not all matrices are invertible (even nonzero matrices can be
  singular).
\item Matrix multiplication is associative, but not commutative.
\end{itemize}
Don't forget these facts!

In matrix computations, we deal not only with the linear algebraic
perspective on a matrix, but also with concrete representations.  We
usually represent a dense matrix as an array of numbers that are
stored sequentially in computer memory.  But we may use different
representations depending on what we want to do.  Often our goal is to
evaluate some expression involving a matrix, such as evaluating a
linear map or a quadratic form or solving a linear system.  In these
cases, we might prefer other different representations that take
advantage of a particular problem's structure.

\section{Twelve Commandments}

When Charlie Van Loan teaches matrix computations, he states
``twelve commandments'' of matrix manipulations:
\begin{enumerate}
\item Matrix $\times$ vector $=$ linear combination of matrix columns.
\item Inner product $=$ sum of products of corresponding elements.
\item Order of operations is important to performance.
\item Matrix $\times$ diagonal $=$ scaling of the matrix columns.
\item Diagonal $\times$ matrix $=$ scaling of the matrix rows.
\item Never form an explicit diagonal matrix.
\item Never form an explicit rank one matrix.
\item Matrix $\times$ matrix $=$ collection of matrix-vector products.
\item Matrix $\times$ matrix $=$ collection of dot products.
\item Matrix $\times$ matrix $=$ sum of rank one matrices.
\item Matrix $\times$ matrix $\implies$ linearly combine rows from
  the second matrix.
\item Matrix $\times$ matrix $\implies$ linearly combine columns from
  the first matrix.
\end{enumerate}
I might add more, but twelve is a nicer-sounding
number than thirteen or fourteen, and fifteen is clearly too many.

\section{Block matrices}

We often partition matrices into submatrices of different
sizes.  For example, we might write
\[
  \begin{bmatrix}
    a_{11} & a_{12} & b_1 \\
    a_{21} & a_{22} & b_2 \\
    c_1 & c_2 & d
  \end{bmatrix} =
  \begin{bmatrix}
    A & b \\
    c^T & d
  \end{bmatrix}, \mbox{ where }
  A = \begin{bmatrix} a_{11} & a_{12} \\ a_{21} & a_{22} \end{bmatrix},
  b = \begin{bmatrix} b_1 \\ b_2 \end{bmatrix},
  c = \begin{bmatrix} c_1 \\ c_2 \end{bmatrix}.
\]
We can manipulate block matrices in much the same way we manipulate
ordinary matrices; we just need to remember that matrix multiplication
does not commute.

\section*{Standard matrices}

We will see a handful of standard matrices throughout the
course:
\begin{itemize}
\item The zero matrix (written $0$ -- we distinguish from the scalar
  zero by context).  In MATLAB: {\tt zeros(m,n)}.
\item The identity matrix $I$.  In MATLAB: {\tt eye(n)}.
\item Diagonal matrices, usually written $D$.  In MATLAB:
  {\tt diag(d)} where {\tt d} is the vector of diagonal entries.
\item Permutation matrices, usually written $P$ or $\Pi$ (but
  sometimes written $Q$ if $P$ is already used).  These are square 0-1
  matrices with exactly one 1 in each row and column.  They look like
  the identity matrix with the columns permuted.  In MATLAB, I would
  usually write {\tt P = eye(n); P = P(:,idx)} where {\tt idx} is an
  index vector such that data at index {\tt idx(k)} in a vector $v$
  gets mapped to index {\tt k} in $Pv$.
\end{itemize}
Though I have given MATLAB commands to construct these matrices,
we usually would not actually create them explicitly except as a
step in creating another matrix (see Van Loan's sixth commandment!).

\section{Matrix shapes and structures}

In linear algebra, we talk about different matrix structures.
For example:
\begin{itemize}
\item $A \in \bbR^{n \times n}$ is {\em nonsingular} if the inverse
  exists; otherwise it is {\em singular}.
\item $Q \in \bbR^{n \times n}$ is {\em orthogonal} if $Q^T Q = I$.
\item $A \in \bbR^{n \times n}$ is {\em symmetric} if $A = A^T$.
\item $S \in \bbR^{n \times n}$ is {\em skew-symmetric} if $S = -S^T$.
\item $L \in \bbR^{n \times m}$ is {\em low rank} if $L = UV^T$
  for $U \in \bbR^{n \times k}$ and $V \in \bbR^{m \times k}$ where
  $k \ll \min(m,n)$.
\end{itemize}
These are properties of an underlying linear map or quadratic form; if
we write a different matrix associated with an (appropriately
restricted) change of basis, it will also have the same properties.

In matrix computations, we also talk about the {\em shape} (nonzero
structure) of a matrix.  For example:
\begin{itemize}
\item $D$ is {\em diagonal} if $d_{ij} = 0$ for $i \neq j$.
\item $T$ is {\em tridiagonal} if $t_{ij} = 0$ for $i \not \in \{j-1,
  j, j+1\}$.
\item $U$ is {\em upper triangular} if $u_{ij} = 0$ for $i > j$
  and {\em strictly upper triangular} if $u_{ij} = 0$ for $i \geq j$
  (lower triangular and strictly lower triangular are similarly
  defined).
\item $H$ is {\em upper Hessenberg} if $h_{ij} = 0$ for $i > j+1$.
\item $B$ is {\em banded} if $b_{ij} = 0$ for $|i-j| > \beta$.
\item $S$ is {\em sparse} if most of the entries are zero.  The
  position of the nonzero entries in the matrix is called the
  {\em sparsity structure}.
\end{itemize}
We often represent the shape of a matrix by marking where the nonzero
elements are (usually leaving empty space for the zero elements); for
example:
\begin{align*}
  \mbox{Diagonal} &
  \begin{bmatrix}
    \times & & & & \\
    & \times & & & \\
    & & \times & & \\
    & & & \times & \\
    & & & & \times
  \end{bmatrix} &
  \mbox{Tridiagonal} &
  \begin{bmatrix}
    \times & \times & & & \\
    \times & \times & \times & & \\
    & \times & \times & \times & \\
    & & \times & \times & \times \\
    & & & \times & \times
  \end{bmatrix} \\
  \mbox{Triangular} &
  \begin{bmatrix}
    \times & \times & \times & \times & \times \\
    & \times & \times & \times & \times \\
    & & \times & \times & \times \\
    & & & \times & \times \\
    & & & & \times
  \end{bmatrix} &
  \mbox{Hessenberg} &
  \begin{bmatrix}
    \times & \times & \times & \times & \times \\
    \times & \times & \times & \times & \times \\
    & \times & \times & \times & \times \\
    & & \times & \times & \times \\
    & & & \times & \times
  \end{bmatrix} \\
\end{align*}
We also sometimes talk about the {\em graph} of a (square) matrix
$A \in \bbR^{n \times n}$: if we assign a node to each index
$\{1, \ldots, n\}$, an edge $(i,j)$ in the graph corresponds
to $a_{ij} \neq 0$.  There is a close connection between certain
classes of graph algorithms and algorithms for factoring sparse
matrices or working with different matrix shapes.  For example,
the matrix $A$ can be permuted so that $P A P^T$ is upper triangular
iff the associated directed graph is acyclic.

The shape of a matrix (or graph of a matrix) is not intrinsically
associated with a more abstract linear algebra concept; apart from
permutations, sometimes, almost any change of basis will completely
destroy the shape.

\section{Data sparsity and fast matrix-vector products}

We say a matrix $A \in \bbR^{n \times m}$ is {\em data sparse} if we
can represent it with far fewer than $nm$ parameters.  For example,
\begin{itemize}
\item Sparse matrices are data sparse -- we only need to explicitly
  know the positions and values of the nonzero elements.
\item A rank one matrix is data sparse: if we write it as an outer
  product $A = uv^T$, we need only $n+m$ parameters (we can actually
  get away with only $n+m-1$, but usually wouldn't bother).  More
  generally, low-rank matrices are data sparse.
\item A Toeplitz matrix (constant diagonal entries) is data sparse.
\item The upper or lower triangle of a low rank matrix is data sparse.
\end{itemize}
Sums and products of a few data sparse matrices
will remain data sparse.

Data sparsity is useful for several reasons.  If we are interested
in the matrix itself, data sparsity lets us save storage.  If we are
interested in multiplying by the matrix, or solving linear systems
involving the matrix, data sparsity lets us write fast algorithms.
For example,
\begin{itemize}
\item We can multiply a sparse matrix $A$ times a vector in
  $O(\mbox{nnz})$ time in general, where nnz is the number of
  nonzeros in the matrix.
\item If $A = uv^T$ is rank one, we can compute $y = Ax$ in
  $O(n+m)$ time by first computing $\alpha = v^T x$ (a dot product, $O(m)$
  time), then $y = \alpha u$ (a scaling of the vector $u$, $O(n)$
  time).
\item We can multiply a square Toeplitz matrix by a vector in
  $O(n \log n)$ time using fast Fourier transforms.
\item We can multiply the upper or lower triangle of a square low rank
  matrix by a vector in $O(n)$ time with a simple loop (left as an
  exercise).
\end{itemize}
In much modern research on numerical linear algebra, sparsity or
data sparsity is the name of the game.  Few large problems are
unstructured; and where there is structure, one can usually play
games with data sparsity.

\end{document}

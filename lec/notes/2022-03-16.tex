\documentclass[12pt, leqno]{article}
\usepackage{fancyhdr}
\usepackage[letterpaper=true,colorlinks=true,linkcolor=black]{hyperref}

\usepackage{amsfonts}
\usepackage{amsmath}
\usepackage{amssymb}
\usepackage{color}
\usepackage{tikz}
\usepackage{pgfplots}
\usepackage{listings}
%\usepackage{courier}
%\usepackage[utf8]{inputenc}
%\usepackage[russian]{babel}

\lstdefinelanguage{Julia}%
  {morekeywords={abstract,break,case,catch,const,continue,do,else,elseif,%
      end,export,false,for,function,immutable,import,importall,if,in,%
      macro,module,otherwise,quote,return,switch,true,try,type,typealias,%
      using,while},%
   sensitive=true,%
   alsoother={$},%
   morecomment=[l]\#,%
   morecomment=[n]{\#=}{=\#},%
   morestring=[s]{"}{"},%
   morestring=[m]{'}{'},%
}[keywords,comments,strings]%

\lstset{
  numbers=left,
  basicstyle=\ttfamily\footnotesize,
  numberstyle=\tiny\color{gray},
  stepnumber=1,
  numbersep=10pt,
}

\newcommand{\iu}{\ensuremath{\mathrm{i}}}
\newcommand{\bbR}{\mathbb{R}}
\newcommand{\bbC}{\mathbb{C}}
\newcommand{\calV}{\mathcal{V}}
\newcommand{\calE}{\mathcal{E}}
\newcommand{\calG}{\mathcal{G}}
\newcommand{\calW}{\mathcal{W}}
\newcommand{\calP}{\mathcal{P}}
\newcommand{\macheps}{\epsilon_{\mathrm{mach}}}
\newcommand{\matlab}{\textsc{Matlab}}
\newcommand{\uQ}{\underline{Q}}
\newcommand{\uR}{\underline{R}}

\newcommand{\ddiag}{\operatorname{diag}}
\newcommand{\fl}{\operatorname{fl}}
\newcommand{\nnz}{\operatorname{nnz}}
\newcommand{\tr}{\operatorname{tr}}
\renewcommand{\vec}{\operatorname{vec}}

\newcommand{\vertiii}[1]{{\left\vert\kern-0.25ex\left\vert\kern-0.25ex\left\vert #1
    \right\vert\kern-0.25ex\right\vert\kern-0.25ex\right\vert}}
\newcommand{\ip}[2]{\langle #1, #2 \rangle}
\newcommand{\ipx}[2]{\left\langle #1, #2 \right\rangle}
\newcommand{\order}[1]{O( #1 )}

\newcommand{\kron}{\otimes}


\newcommand{\hdr}[1]{
  \pagestyle{fancy}
  \lhead{Bindel, Spring 2022}
  \rhead{Numerical Analysis}
  \fancyfoot{}
  \begin{center}
    {\large{\bf #1}}
  \end{center}
  \lstset{language=Julia,columns=flexible}  
}


\newcommand{\calK}{\mathcal{K}}
%\newcommand{\calP}{\mathcal{P}}
\newcommand{\calR}{\mathcal{R}}

\begin{document}
\hdr{2022-03-16}

\section{Linear Solves and Quadratic Minimization}

We have already briefly described an argument that Jacobi iteration
converges for strictly row diagonally dominant matrices.  We now
discuss an argument that Gauss-Seidel converges (or at least part of
such an argument).  In the process, we will see a useful way of
reformulating the solution of symmetric positive definite linear
systems that will prepare us for our upcoming discussion of conjugate
gradient methods.

Let $A$ be a symmetric positive definite matrix, and consider
the ``energy'' function
\[
  \phi(x) = \frac{1}{2} x^T A x - x^T b.
\]
The stationary point for this function is the point at which
the derivative in any direction is zero.  That is, for any
direction vector $u$,
\begin{align*}
  0
  &=\left. \frac{d}{d\epsilon} \right|_{\epsilon = 0} \phi(x+\epsilon u) \\
  &=\frac{1}{2} u^T A x + \frac{1}{2} x^T A u -  u^T b \\
  &=u^T (Ax-b)
\end{align*}
Except in pathological instances, a directional derivative can be
written as the dot product of a direction vector and a gradient;
in this case, we have
\[
  \nabla \phi = Ax-b.
\]
Hence, minimizing $\phi$ is equivalent to solving $Ax = b$~\footnote{%
If you are unconvinced that this is a minimum, work through the
algebra to show that $\phi(A^{-1} b + w) = \frac{1}{2} w^T A w$ for any $w$.}.

Now that we have a characterization of the solution of $Ax = b$
in terms of an optimization problem, what can we do with it?
One simple approach is to think of a sweep through all the unknowns,
adjusting each variable in term to minimize the energy; that is,
we compute a correction $\Delta x_j$ to node $j$ such that
\[
  \Delta x_j = \operatorname{argmin}_z \phi(x+z e_j)
\]
Note that
\[
  \frac{d}{dz} \phi(x+z e_j) = e_j^T (A(x+ze_j)-b),
\]
and the update $x_j := x_j + \Delta x_j$ is equivalent to choosing
a new $x_j$ to set this derivative equal to zero.  But this is
exactly what the Gauss-Seidel update does!  Hence, we can see
Gauss-Seidel in two different ways: as a stationary method for solving
a linear system, or as an optimization method that constantly makes
progress toward a solution that minimizes the energy~\footnote{%
Later in the class, we'll see this as coordinate-descent with
exact line search.}.
The latter perspective can be turned (with a little work) into
a convergence proof for Gauss-Seidel on positive-definite linear systems.

\section{Extrapolation: A Hint of Things to Come}

Stationary iterations are simple.  Methods like Jacobi or Gauss-Seidel
are easy to program, and it's (relatively) easy to analyze their
convergence.  But these methods are also often slow.  We'll talk next
time about more powerful {\em Krylov subspace} methods that use
stationary iterations as a building block.

There are many ways to motivate Krylov subspace methods.  We'll
pick one motivating idea that extends beyond
the land of linear solvers and into other applications as well.
The key to this idea is the observation that the error in our
iteration follows a simple pattern:
\[
  x^{(k)}-x = e^{(k)} = R^k e^{(0)}, \quad R = M^{-1} N.
\]
Suppose $R$ is diagonalizable, i.e. $R = V \Lambda V^{-1}$, and let
$V^{-1} e^{(0)} = c$.  Then we have
\[
  e^{(k)} = V \Lambda^{k} V^{-1} e^{(0)} = V \Lambda^k c
          = \sum_{j=1}^n v_j \lambda_j^k c_j.
\]
Assuming there is a unique dominant eigenvalue, the behavior of the
error is dominated by that eigenvalue for large $k$, i.e.
\[
  e^{(k+1)} \approx \lambda_1 e^{(k)}.
\]
Note that this means
\[
  x^{(k+1)}-x^{(k)} = e^{(k+1)}-e^{(k)} \approx (\lambda_1-1) e^{(k)}.
\]
If we have an estimate for $\lambda_1$, we can write
\[
  x = x^{(k)} - e^{(k)} \approx
  x^{(k)}-\frac{x^{(k+1)}-x^{(k)}}{\lambda_1-1}.
\]
That is, we might hope to get a better estimate of $x$ than is
provided by $x^{(k)}$ or $x^{(k+1)}$ individually by taking an
appropriate linear combination of $x^{(k)}$ and $x^{(k+1)}$.

How might we get an estimate for $\lambda_1$?  In some cases,
we might be able to guess a good estimate from other context.
Otherwise, though, we might estimate $\lambda_1$ via a Rayleigh
quotient.  Let $u^{(k+1)} = x^{(k+1)}-x^{(k)}$; we know from our
iteration equation that $u^{(k+1)} = R u^{(k)}$, so we might try
to use Rayleigh quotients to estimate $\lambda_1$:
\[
  \rho_R(u^{(k)})
    = \frac{u^{(k)} \cdot R u^{(k)}}{\|u^{(k)}\|^2}
    = \frac{u^{(k)} \cdot u^{(k+1)}}{\|u^{(k)}\|^2}.
\]
Plugging this into our estimate for the error correction gives us
a transformed sequence
\[
  \tilde{x}^{(k)} = x^{(k)} - \frac{u^{(k)}{\rho_R(u^{(k)})-1},
\]
which generally converges to the true solution faster than the
original sequence converged.

This idea generalizes: if we have a sequence of approximations
$x^{(0)}, \ldots, x^{(k)}$, why not ask for the ``best'' approximation
that can be written as a linear combination of the $x^{(j)}$?  This is
the notion underlying Krylov methods, which we will discuss next time.

\section{An Aside on Extrapolation}

Many students are only exposed to extrapolation methods as a trick or
an aside in a numerical methods course.  This really does not do
justice to a family of methods with a deep history and connections
across the breadth of mathematics.  Extrapolation methods are not only
a sometimes-miraculous-seeming tool in numerics, but they are
intimately connected to continued fractions and rational approximation
(itself an under-studied area in modern times!), to questions in
number theory (e.g. the proofs that $e$ and $\pi$ are transcendental),
to time series analysis, to filtering and signal processing, and to
many other pure and applied topics.  They have been re-invented
repeatedly across areas and cultures.  Recent literature often names
the methods after 20th-century mathematicians, but the ideas go back
at least to the 17th century, and not just in Europe (what is now
called Aitken's delta-squared process was known to 17th century
Japanese mathematics as a way of accelerating series converging to
$\pi$).  I really like the
\href{https://doi.org/10.1016/0168-9274(95)00110-7}{historical survey by Brezinski}
for some of these connections and background.


The earliest extrapolation methods were seen as sequence
transformations, converting one sequence into another sequence with
faster convergence.  Indeed, sometimes these methods convert divergent
sequences into convergent ones, giving us the notion of an
``anti-limit!''  But I like to think of a hierarchy of methods that one
can use depending on how much additional context one has:
\begin{itemize}
\item
  Standard extrapolation methods convert one sequence $x_1, x_2, \ldots$
  to a new sequence $\check{x}_1, \check{x}_2, \ldots$ that converges
  more quickly.  The transformation usually involves a model for the
  error that is fit to successive entries in the original sequence.  But
  otherwise, we don't explicitly use anything about how the sequence is
  generated.
\item Acceleration or mixing procedures (like
  \href{https://epubs.siam.org/doi/10.1137/10078356X}{Anderson acceleration}
  or
  \href{https://en.wikipedia.org/wiki/DIIS}{Pulay mixing})
  explicitly assume that we have a fixed point iteration
  $x^{(k+1)} = G(x^{(k)})$ with a known function $G$.
  Unlike more general sequence
  extrapolation methods, these methods apply $G$ as part of the
  computation of the new sequence.
\item
  Other methods, including the Krylov subspace methods that we are
  about to describe, are posed in terms of approximately solving a
  system of equations $F(x) = 0$ or minimizing some objective function
  $\phi(x)$.  These methods construct a space of possible
  approximations from linear combinations of the iterates of another
  method, and then use some ansatz to choose the ``best possible''
  approximation from that space.  This ``best possible'' approximation
  might be one that minimizes some residual error $\|F(x)\|$ over the
  subspace, or it might minimize $\phi(x)$, or it might satisfy some
  other condition.
\end{itemize}

Methods that use more detailed knowledge of a system of equations or
optimization problem we want to solve are less general than some of
the classical methods -- they aren't going to help us at all if we
want to accelerate a sequence converging to $\pi$!  But they also tend
to be less numerically delicate than standard extrapolation measure,
which my their nature often involve a lot of cancellation effects in
order to estimate errors from successive steps.

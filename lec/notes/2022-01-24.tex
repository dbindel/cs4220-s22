\documentclass[12pt, leqno]{article}
\usepackage{fancyhdr}
\usepackage[letterpaper=true,colorlinks=true,linkcolor=black]{hyperref}

\usepackage{amsfonts}
\usepackage{amsmath}
\usepackage{amssymb}
\usepackage{color}
\usepackage{tikz}
\usepackage{pgfplots}
\usepackage{listings}
%\usepackage{courier}
%\usepackage[utf8]{inputenc}
%\usepackage[russian]{babel}

\lstdefinelanguage{Julia}%
  {morekeywords={abstract,break,case,catch,const,continue,do,else,elseif,%
      end,export,false,for,function,immutable,import,importall,if,in,%
      macro,module,otherwise,quote,return,switch,true,try,type,typealias,%
      using,while},%
   sensitive=true,%
   alsoother={$},%
   morecomment=[l]\#,%
   morecomment=[n]{\#=}{=\#},%
   morestring=[s]{"}{"},%
   morestring=[m]{'}{'},%
}[keywords,comments,strings]%

\lstset{
  numbers=left,
  basicstyle=\ttfamily\footnotesize,
  numberstyle=\tiny\color{gray},
  stepnumber=1,
  numbersep=10pt,
}

\newcommand{\iu}{\ensuremath{\mathrm{i}}}
\newcommand{\bbR}{\mathbb{R}}
\newcommand{\bbC}{\mathbb{C}}
\newcommand{\calV}{\mathcal{V}}
\newcommand{\calE}{\mathcal{E}}
\newcommand{\calG}{\mathcal{G}}
\newcommand{\calW}{\mathcal{W}}
\newcommand{\calP}{\mathcal{P}}
\newcommand{\macheps}{\epsilon_{\mathrm{mach}}}
\newcommand{\matlab}{\textsc{Matlab}}
\newcommand{\uQ}{\underline{Q}}
\newcommand{\uR}{\underline{R}}

\newcommand{\ddiag}{\operatorname{diag}}
\newcommand{\fl}{\operatorname{fl}}
\newcommand{\nnz}{\operatorname{nnz}}
\newcommand{\tr}{\operatorname{tr}}
\renewcommand{\vec}{\operatorname{vec}}

\newcommand{\vertiii}[1]{{\left\vert\kern-0.25ex\left\vert\kern-0.25ex\left\vert #1
    \right\vert\kern-0.25ex\right\vert\kern-0.25ex\right\vert}}
\newcommand{\ip}[2]{\langle #1, #2 \rangle}
\newcommand{\ipx}[2]{\left\langle #1, #2 \right\rangle}
\newcommand{\order}[1]{O( #1 )}

\newcommand{\kron}{\otimes}


\newcommand{\hdr}[1]{
  \pagestyle{fancy}
  \lhead{Bindel, Spring 2022}
  \rhead{Numerical Analysis}
  \fancyfoot{}
  \begin{center}
    {\large{\bf #1}}
  \end{center}
  \lstset{language=Julia,columns=flexible}  
}


\begin{document}
\hdr{2022-01-24}

\section{What are we about?}

Welcome to ``Numerical Analysis: Linear and Nonlinear Equations'' (CS
4220, CS 5223, and Math 4260).  This is part of a pair of courses
offered jointly between CS and math that provide an introduction to
scientific computing.  My own tongue-in-cheek summary of scientific
computing is that it is the art of solving problems of continuous
mathematics fast enough and accurately enough.  Of course, what
constitutes ``fast enough'' and ``accurately enough'' depends on
context, and learning to reason about that context is also part of the
class.

Because our survey is partitioned into two semesters, we do not cover
all the standard topics in a single semester.  In particular, this class
will (mostly) not cover interpolation and function approximation,
numerical differentiation and quadrature, or the solution of ordinary
and partial differential equations.  We will focus instead on numerical
linear algebra, nonlinear equation solving, and optimization.  Broadly
speaking, we will spend the first half of the semester on {\em
factorization} methods for linear algebra problems, and the latter half
on {\em iterative} methods for both linear and nonlinear problems.  As
currently planned, the schedule also includes time for one special topic
week that I expect will bring together several of the themes from the
course.

\subsection{Learning outcomes}

We are going to cover a variety of methods for a variety of problems
by the end of the semester.  But the goal is that by the end of the
semester, you will be able to
\begin{itemize}
\item {\bf Analyze sources of error} in numerical algorithms and reason about
  problem stability and how it influences the accuracy of numerical
  computations.
\item {\bf Choose appropriate numerical algorithms to solve linear
  algebra problems} (linear systems, least squares problems, and
  eigenvalue problems) taking into account problem structure.
\item {\bf Formulate nonlinear equations and constrained and
  unconstrained optimization problems} for solution on a computer.
\item {\bf Analyze the local convergence} of nonlinear solver
  algorithms.
\item {\bf Reason about global convergence} of nonlinear solver
  algorithms.
\item {\bf Use numerical methods} to solve problems of practical
  interest from data science, machine learning, and engineering.
\end{itemize}

\subsection{Mathematics, Computation, Application}

Our focus will be the mathematical and computational structure of
numerical methods.  But we use numerical methods to solve
problems from applications, and a scientific computing class with
no applications is far less rich and interesting than it ought to be.
So we will, when possible, try to bring in application examples.

The majority of students in the class come from computer science.  We
also have students from a wide variety of other majors.  This means
that students come to the class with different levels of background
and interest in a variety of application domains.  Because of the
nature of the enrollment, many of my examples will come from areas
conventionally associated with computer science and mathematics, but
there will also be the odd example from physics or engineering.  So if
we dig into an application problem and you get lost, don't worry -- I
don't expect you to know this already!  Also, ask questions, as there
are bound to be others the class who are equally confused.

\subsection{Cross-cutting themes}

There are some themes that cut across topics in the syllabus,
and I expect we will touch on these themes frequently through
the semester.  These include:
\begin{itemize}
\item {\bf Knowing the answer in advance} -- It's dangerous to go into
  a computation with no idea what to expect.  The structure of the
  problem and the solution affect how we choose methods and how we
  evaluate success.  A qualitative analysis or ballpark estimate of
  solution behavior is usually the first step to intelligently
  applying a numerical method.
\item {\bf Pictures and plots} -- Careful pictures tell us a lot.
  Plot an approximate solution.  Are there unexpected oscillations or
  negative values, or crazy-looking behaviors near the domain of the
  soution?  Maybe you should investigate!  Similarly, plots of error
  with respect to a spatial variable or a step number often provide
  key insights into whether a method is working as desired.
\item {\bf Documentation, testing, and error checking} -- When we
  write numerical codes, the implied agreement between the author of
  the code and the user of the code is often more subtle than
  the agreements behind other software interfaces.  Call a sort
  routine, and it will sort your data in some specified time.  Call a
  linear solver, and it will solve your problem in an amount of time
  that depends on the problem structure and with a level of accuracy
  that depends on the problem characteristics.  This makes good
  software hygiene -- careful documentation, testing, error
  checking, and design for reproducibility -- both tricky and important!
\item {\bf Modularity and composability} -- When we compose numerical
  methods, we have to worry about error.  Even if you expect only
  to use numerical building blocks (and never build them yourself), it
  is important to understand the types of error and performance
  guarantees one can make and how they are useful in reasoning about
  large computational codes.
\item {\bf Problem formulation and choice of representation} -- Often, the
  same problem can be posed in many different ways.  Some suggest
  simple, efficient numerical methods.  Others are impossibly hard.
  The key difference between the two is often in how we represent the
  problem data and the thing we seek.
\item {\bf Numerical anti-patterns} -- Some operations, such as
  computing explicit inverses and determinants, are perfectly natural
  in symbolic mathematics but turn out to be terrible ideas in
  numerical computations.  We will point these out as we come across
  them.
\item {\bf Time and memory scalability} -- We often want to solve
  big problems, and it is important to understand before we start
  whether we think we can solve a problem on a laptop in a second or
  two or if we really need a month on a supercomputer.  This means
  we would like a rough estimate -- usually posed in terms of order notation
  -- of the time and memory complexity of different algorithms.
\item {\bf Blocking and building with high-performance blocks} --
  Building fast codes is hard.  As numerical problem solvers, we would
  like someone else to do much of this hard work so that we can focus
  on other things.  This means we need to understand the common
  building blocks, and a little bit about not only their complexity,
  but also why they are fast or slow on real machines.
\item {\bf Performance tradeoffs in iterations} -- Iterative methods
  produce a sequence of approximate solutions that (one hopes) get
  closer and closer to the right answer.  To choose iterations
  intelligently, we need to understand the tradeoffs between the
  time to compute an iteration, the progress that one can make, and
  the overall stability of an iterative procedure.
\item {\bf Convergence monitoring and stopping} -- One of the hardest
  parts of designing an iterative method is often deciding when to
  stop.  This point will recur several times in the second half of the
  semester.
\item {\bf Use of approximations and surrogates} -- Simple surrogate
  models are an important part of the design of nonlinear iterations.
  We will be particularly interested in local polynomial
  approximations, but we may talk about some others as well.
\end{itemize}

\section{Logistics}

We will go through the syllabus in detail, but at a high level you
should plan on six homeworks (individual) and three projects (in
pairs), a midterm, and a final.  If you are taking this course as
5223, you will also have a semester project involving choosing and
implementing a numerical method of choice from the literature.  I will
also ask you for feedback at the middle and end of the semester, and
this counts for credit.

Another 10\% of your grade involves in-class work: plan to bring a
sheet of paper to turn in, with each in-class submission worth a third
of a point (there are 42 lectures, so you can miss a few without
penalty).

Homework and projects are due via CMS by midnight on Fridays; we allow
some ``slip days'' so that you can work on an assignment through the
weekend if needed.  We will also drop the lowest of the HW grades, in
case there is a particularly hectic week.  Office hours are TBD,
but we will announce them soon.  You can also request office
hours by appointment.

\subsection{Infrastructure}

The first two weeks of classes (and office hours) are on
Zoom.  We will keep the Zoom link active for the rest of the semester
in case we return to online instruction, whether because of COVID or
weather, as well as providing a way to provide lectures if I need to
travel.

Class notes and assignments, as well as class announcements, will be
posted on the course home page.  For submissions, solutions, and
grades, we will use the CS Course Management System (CMS) software.
For class discussion, we will use Ed Discussions.  Canvas is set up
for the class, but is really only there for the Zoom setup.  There are
links from each of these pages to the others; I recommend you use the
class web page as your starting point.

We will use Julia in our notes, and you should also use Julia for your
homework.  I am using the most recent version (Julia 1.7), and
recommend you use the same.  I recommend installing Julia on your own
machine, but you are also welcome to use an online service like JuliaHub.

The course web page is maintained from a repository on GitHub.
I encourage you to submit corrections or enhancements by pull
request!

\subsection{Background}

The formal prerequisites for the class are linear algebra at the level
of Math 2210 or 2940 or equivalent and a CS 1 course in any language.
We also recommend one additional math course at the 3000 level or
above; this is essentially a proxy for ``sufficient mathematical
maturity.''

In practice: I will assume you know some multivariable calculus
and linear algebra, and that your CS background includes not only
basic programming but also some associated mathematical concepts
(e.g.~order notation and a little graph theory).  If you feel your
background is weak in these areas, please talk to us.

Some of you may want to review your linear algebra basics in particular.
At Cornell, our undergraduate linear algebra course uses the text
by Lay~\cite{Lay:2016:Linear}; the texts by Strang~\cite{Strang:2006:Linear,Strang:2009:Introduction} are a nice
alternative.  Strang's {\em Introduction to Linear Algebra}~\cite{Strang:2009:Introduction} is the textbook for the MIT
linear algebra course that is the basis for his enormously popular
video lectures, available on MIT's OpenCourseWare site; if you prefer
lecture to reading, Strang is known as an excellent lecturer.

\section{Basic notational conventions}

In this section, we set out some basic notational conventions used
in the class.
\begin{enumerate}
\item
  The complex unit is $\iu$ (not $i$ or $j$).

\item
  By default, all spaces in this class are finite dimensional. If there
  is only one space and the dimension is not otherwise stated, we use
  $n$ to denote the dimension.

\item
  Concrete real and complex vector spaces are $\bbR^n$ and $\bbC^n$,
  respectively.

\item
  Real and complex matrix spaces are $\bbR^{m\times n}$ and $\bbC^{m
  \times n}$.

\item Unless otherwise stated, a concrete vector is a column vector.

\item The vector $e$ is the vector of all ones.

\item The vector $e_i$ has all zeros except a one in the $i$th place.

\item
  The basis $\{ e_i \}_{i=1}^n$ is the {\em standard basis} in $\bbR^n$
  or $\bbC^n$.

\item
  We use calligraphic math caps for abstract space,
  e.g.~$\mathcal{U}, \mathcal{V}, \mathcal{W}$.

\item
   When we say $U$ is a basis for a space $\mathcal{U}$, we mean $U$ is
   an isomorphism $\mathcal{U} \rightarrow \bbR^n$.  By a slight abuse
   of notation, we say $U$ is a matrix whose columns are the abstract
   vectors $u_1, \ldots, u_n$, and we write the linear combination
   $\sum_{i=1}^n u_i c_i$ concisely as $Uc$.

\item
  Similarly, $U^{-1} x$ represents the linear mapping from the
  abstract vector $x$ to a concrete coefficient vector $c$ such
  that $x = Uc$.

\item The space of univariate polynomials of degree at most
  $d$ is $\mathcal{P}_d$.

\item
  Scalars will typically be lower case Greek, e.g.~$\alpha, \beta$. In
  some cases, we will also use lower case Roman letters, e.g.~$c, d$.

\item
  Vectors (concrete or abstract) are denoted by lower case Roman,
  e.g.~$x, y, z$.

\item
  Matrices and linear maps are both denoted by upper case Roman,
  e.g.~$A, B, C$.

\item
  For $A \in \bbR^{m \times n}$, we denote the entry in row $i$ and
  column $j$ by $a_{ij}$.  We reserve the notation $A_{ij}$ to refer to
  a submatrix at block row $i$ and block column $j$ in a partitioning of
  $A$.

\item
  We use a superscript star to denote dual spaces and dual vectors; that
  is, $v^* \in \mathcal{V}^*$ is a dual vector in the space dual to
  $\mathcal{V}$.

\item
  In $\bbR^n$, we use $x^*$ and $x^T$ interchangeably for the transpose.

\item
  In $\bbC^n$, we use $x^*$ and $x^H$ interchangeably for the conjugate
  transpose.

\item
  Inner products are denoted by angles, e.g.~$\ip{x}{y}$. To denote an
  alternate inner product, we use subscripts, e.g. $\ip{x}{y}_M = y^* M
  x$.

\item
  The standard inner product in $\bbR^n$ or $\bbC^n$ is also $x \cdot y$.

\item
  In abstract vector spaces with a standard inner product, we use $v^*$
  to denote the dual vector associated with $v$ through the inner
  product, i.e.~$v^* = (w \mapsto \ip{w}{v})$.

\item
  We use the notation $\|x\|$ to denote a norm of the vector $x$. As
  with inner products, we use subscripts to distinguish between multiple
  norms.  When dealing with two generic norms, we will sometimes use the
  notation $\vertiii{y}$ to distinguish the second norm from the first.

\item
  We use order notation for both algorithm scaling with parameters going
  to infinity (e.g.~$\order{n^3}$ time) and for reasoning about scaling
  with parameters going to zero (e.g.~$\order{\epsilon^2}$ error). We
  will rely on context to distinguish between the two.

\item
  We use {\em variational notation} to denote derivatives of matrix
  expressions, e.g.~$\delta (AB) = \delta A \, B + A \, \delta B$ where
  $\delta A$ and $\delta B$ represent infinitesimal changes to the
  matrices $A$ and $B$.

\item
  Symbols typeset in Courier font should be interpreted as \matlab\ or
  Julia code or pseudocode, e.g.~{\tt y = A*x}.

\item
  The function notation $\fl(x)$ refers to taking a real or complex
  quantity (scalar or vector) and representing each entry in floating
  point.

\end{enumerate}

\section{Matrix algebra versus linear algebra}

\begin{quotation}
  We share a philosophy about linear algebra: we think basis-free, we write basis-free, but when the chips are down we close the office door and compute with matrices like fury. \\
  \hspace*{\fill} --- Irving Kaplansky
  on the late Paul Halmos~\cite{Ewing:1991:Halmos},
\end{quotation}

Linear algebra is fundamentally about the structure of vector spaces
and linear maps between them.  A matrix represents a linear map with
respect to some bases.  Properties of the underlying linear map may
be more or less obvious via the matrix representation associated with
a particular basis, and much of matrix computations is about finding
the right basis (or bases) to make the properties of some linear map
obvious.  We also care about finding changes of basis that are ``nice''
for numerical work.

In some cases, we care not only about the linear map a matrix
represents, but about the matrix itself.  For example, the {\em graph}
associated with a matrix $A \in \bbR^{n \times n}$ has vertices $\{1,
\ldots, n\}$ and an edge $(i,j)$ if $a_{ij} \neq 0$.  Many of the
matrices we encounter in this class are special because of the structure
of the associated graph, which we usually interpret as the ``shape'' of
a matrix (diagonal, tridiagonal, upper triangular, etc).  This structure
is a property of the matrix, and not the underlying linear
transformation; change the bases in an arbitrary way, and the graph
changes completely.  But identifying and using special graph structures
or matrix shapes is key to building efficient numerical methods for all
the major problems in numerical linear algebra.

In writing, we represent a matrix concretely as an array of numbers.
Inside the computer, a {\em dense} matrix representation is a
two-dimensional array data structure, usually ordered row-by-row or
column-by-column in order to accomodate the one-dimensional structure of
computer memory address spaces.  While much of our work in the class
will involve dense matrix layouts, it is important to realize that there
are other data structures!  The ``best'' representation for a matrix
depends on the structure of the matrix and on what we want to do with
it.  For example, many of the algorithms we will discuss later in the
course only require a black box function to multiply an (abstract)
matrix by a vector.

\section{Dense matrix basics}

% Level 1-2 BLAS, locality, blocking ideas, memory layout
% Asides regarding JIT compilation in MATLAB

There is one common data structure for dense vectors: we store
the vector as a sequential array of memory cells.  In contrast,
there are {\em two} common data structures for general dense matrices.
In Julia (and MATLAB, NumPy, Fortran), matrices are stored
in {\em column-major} form.
For example, an array of the first four positive integers interpreted
as a two-by-two column major matrix represents the matrix
\[
    \begin{bmatrix} 1 & 3 \\ 2 & 4 \end{bmatrix}.
\]
The same array, when interpreted as a {\em row-major} matrix, represents
\[
    \begin{bmatrix} 1 & 2 \\ 3 & 4 \end{bmatrix}.
\]
Unless otherwise stated, we will assume all dense matrices are represented
in column-major form for this class.  As we will see, this has some
concrete effects on the efficiency of different types of algorithms.

\subsection{The BLAS}

The {\em Basic Linear Algebra Subroutines} (BLAS) are a standard library
interface for manipulating dense vectors and matrices.  There are three
{\em levels} of BLAS routines:
\begin{itemize}
\item {\bf Level 1}:
  These routines act on vectors, and include operations
  such scaling and dot products.  For vectors of length $n$,
  they take $O(n^1)$ time.
\item {\bf Level 2:}
  These routines act on a matrix and a vector, and include operations
  such as matrix-vector multiplication and solution of triangular systems
  of equations by back-substitution.  For $n \times n$ matrices and length
  $n$ vectors, they take $O(n^2)$ time.
\item {\bf Level 3:}
  These routines act on pairs of matrices, and include operations such
  as matrix-matrix multiplication.  For $n \times n$ matrices, they
  take $O(n^3)$ time.
\end{itemize}
All of the BLAS routines are superficially equivalent to algorithms
that can be written with a few lines of code involving one, two, or
three nested loops (depending on the level of the routine).  Indeed,
except for some refinements involving error checking and scaling for
numerical stability, the reference BLAS implementations involve
nothing more than these basic loop nests.  But this simplicity is
deceptive --- a surprising amount of work goes into producing high
performance implementations.

\subsection{Locality and memory}

When we analyze algorithms, we often reason about their complexity
abstractly, in terms of the scaling of the number of operations required
as a function of problem size.  In numerical algorithms, we typically
measure {\em flops} (short for floating point operations).  For example,
consider the loop to compute the dot product of two vectors:
\begin{lstlisting}
function mydot(x, y)
  sum = 0
  for i = 1:length(x)
    sum += x[i]*y[i]
  end
  return sum
end
\end{lstlisting} % DSB
Because it takes $n$ additions and $n$ multiplications, we say this code
takes $2n$ flops, or (a little more crudely) $O(n)$ flops.

On modern machines, though, counting flops is at best a crude way
to reason about how run times scale with problem size.  This is because
in many computations, the time to do arithmetic is dominated by the time
to fetch the data into the processor!  A detailed discussion of modern
memory architectures is beyond the scope of these notes, but there are
at least two basic facts that everyone working with matrix computations
should know:
\begin{itemize}
\item
  Memories are optimized for access patterns with {\em spatial locality}:
  it is faster to access entries of memory that are close to each
  other (ideally in sequential order) than to access memory entries that
  are far apart.  Beyond the memory system, sequential access patterns
  are good for {\em vectorization}, i.e.~for scheduling work to be done
  in parallel on the vector arithmetic units
  that are present on essentially all modern processors.
\item
  Memories are optimized for access patterns with {\em temporal locality};
  that is, it is much faster to access a small amount of data repeatedly
  than to access large amounts of data.
\end{itemize}

The main mechanism for optimizing access patterns with temporal locality
is a system of {\em caches}, fast and (relatively) small memories that can
be accessed more quickly (i.e.~with lower latency) than the main memory.
To effectively use the cache, it is helpful if the {\em working set}
(memory that is repeatedly accessed) is smaller than the cache size.
For level 1 and 2 BLAS routines, the amount of work is proportional to
the amount of memory used, and so it is difficult to take advantage of
the cache.  On the other hand, level 3 BLAS routines do $O(n^3)$ work
with $O(n^2)$ data, and so it is possible for a clever level 3 BLAS
implementation to effectively use the cache.

\subsection{Matrix-vector multiply}

Let us start with two simple Julia programs for matrix-vector
multiplication.  The first one traverses the matrix $A$ one row at a time:
\begin{lstlisting}
function matvec_row(A, x)
  m, n = size(A)
  y = zeros(eltype(x), m)
  for i = 1:m
    for j = 1:n
      y[i] += A[i,j] * x[j]
    end
  end
  return y
end
\end{lstlisting}
The second code traverses a column at a time:
\begin{lstlisting}
function matvec_col(A, x)
  m, n = size(A)
  y = zeros(eltype(x), m)
  for j = 1:n
    for i = 1:m
      y[i] += A[i,j] * x[j]
    end
  end
  return y
end
\end{lstlisting}

It's not too surprising that the builtin matrix-vector multiply routine in
Julia runs faster than either of our hand-written variants, but there
are some other surprises lurking.  We will try timing each of these
matrix-vector multiply methods for random square matrices of
size 4095, 4096, and 4097, to see what happens.  Note that we want to
run each code many times so that we don't get lots of measurement
noise from finite timer granularity; for example,
\begin{lstlisting}
  t1 = @elapsed begin
    for trial = 1:ntrials
      y = A*x
    end
  end  
\end{lstlisting}
In \matlab\ we would do the same thing using {\tt tic} and {\tt toc}.

\begin{figure} \label{fig:matvec-time}
\begin{tikzpicture}
  \begin{axis}[width=\textwidth,xlabel={$n$},ylabel={GFlop/s},xmin=0,ymin=0]

  \addplot table [x index=0, y index=1] {data/matvec_time_jl.dat};
  \addlegendentry{Default};

  \addplot table [x index=0, y index=2] {data/matvec_time_jl.dat};
  \addlegendentry{Row-oriented};

  \addplot table [x index=0, y index=3] {data/matvec_time_jl.dat};
  \addlegendentry{Col-oriented};
  \end{axis}
\end{tikzpicture}  
  \caption{Timing of three matrix-vector multiply implementations.
    In each case, we report the effective time in GFLop/s.
    The line labeled ``default'' is the built-in Julia matvec.}
\end{figure}

On my laptop (a 2021 13 in MacBook Pro with an M1 Pro),
we show the GFlop rates (billions of flops/second)
for the three matrix multiply routines in Figure~\ref{fig:matvec-time}.
There are a few things to notice:
\begin{itemize}
\item The performance of the built-in multiply far exceeds that
  of any of the manual implementations.
\item The peak performance occurs for moderate size matrices where
  the matrix fits into cache, but there is enough work to hide
  the \matlab\ loop overheads.
\item The time required for the built-in routine varies dramatically
  (due to so-called {\em conflict misses}) when the dimension is
  a multiple of a large integer power of two.
\item For $n = 1024$, the column-oriented version (which has good spatial
  locality) is $10\times$ faster than the row-oriented code,
  and $45\times$ faster than the two nested loop version.
\end{itemize}
If you are so inclined, consider yourself encouraged to repeat the
experiment using your favorite compiled language to see if any of the
general trends change significantly.

% Matrix multiplication and blocking
% Memory access and vectorization issues; BLAS routines
% Matrix representation

\subsection{Matrix-matrix multiply}

The classic algorithm to compute $C := C + AB$ involves
three nested loops
\begin{lstlisting}
C = zeros(m,n)
for i = 1:m
  for j = 1:n
    for k = 1:p
      C[i,j] += A[i,k] * B[k,j]
    end
  end
end
\end{lstlisting} % DSB
This is sometimes called an {\em inner product} variant of
the algorithm, because the innermost loop is computing a dot
product between a row of $A$ and a column of $B$.  But
addition is commutative and associative, so we can sum the
terms in a matrix-matrix product in any order and get the same
result.  And we can interpret the orders!  A non-exhaustive
list is:
\begin{itemize}
\item {\tt ij(k)} or {\tt ji(k)}: Compute entry $c_{ij}$ as a
  product of row $i$ from $A$ and column $j$ from $B$
  (the {\em inner product} formulation)
\item {\tt k(ij)}: $C$ is a sum of outer products of column $k$
  of $A$ and row $k$ of $B$ for $k$ from $1$ to $n$
  (the {\em outer product} formulation)
\item {\tt i(jk)} or {\tt i(kj)}: Each row of $C$ is a row of
  $A$ multiplied by $B$
\item {\tt j(ij)} or {\tt j(ki)}: Each column of $C$ is $A$
  multiplied by a column of $C$
\end{itemize}
At this point, we could write down all possible loop orderings
and run a timing experiment, similar to what we did with
matrix-vector multiplication.  But the truth is that high-performance
matrix-matrix multiplication routines use another access pattern
altogether, involving more than three nested loops, and we will
describe this now.

\subsection{Blocking and performance}

The basic matrix multiply outlined in the previous section will
usually be at least an order of magnitude slower than a well-tuned
matrix multiplication routine.  There are several reasons for this
lack of performance, but one of the most important is that the basic
algorithm makes poor use of the {\em cache}.
Modern chips can perform floating point arithmetic operations much
more quickly than they can fetch data from memory; and the way that
the basic algorithm is organized, we spend most of our time reading
from memory rather than actually doing useful computations.
Caches are organized to take advantage of {\em spatial locality},
or use of adjacent memory locations in a short period of program execution;
and {\em temporal locality}, or re-use of the same memory location in a
short period of program execution.  The basic matrix multiply organizations
don't do well with either of these.
A better organization would let us move some data into the cache
and then do a lot of arithmetic with that data.  The key idea behind
this better organization is {\em blocking}.

When we looked at the inner product and outer product organizations
in the previous sections, we really were thinking about partitioning
$A$ and $B$ into rows and columns, respectively.  For the inner product
algorithm, we wrote $A$ in terms of rows and $B$ in terms of columns
\[
  \begin{bmatrix} a_{1,:} \\ a_{2,:} \\ \vdots \\ a_{m,:} \end{bmatrix}
  \begin{bmatrix} b_{:,1} & b_{:,2} & \cdots & b_{:,n} \end{bmatrix},
\]
and for the outer product algorithm, we wrote $A$ in terms of colums
and $B$ in terms of rows
\[
  \begin{bmatrix} a_{:,1} & a_{:,2} & \cdots & a_{:,p} \end{bmatrix}
  \begin{bmatrix} b_{1,:} \\ b_{2,:} \\ \vdots \\ b_{p,:} \end{bmatrix}.
\]
More generally, though, we can think of writing $A$ and $B$ as
{\em block matrices}:
\begin{align*}
  A &=
  \begin{bmatrix}
    A_{11} & A_{12} & \ldots & A_{1,p_b} \\
    A_{21} & A_{22} & \ldots & A_{2,p_b} \\
    \vdots & \vdots &       & \vdots \\
    A_{m_b,1} & A_{m_b,2} & \ldots & A_{m_b,p_b}
  \end{bmatrix} \\
  B &=
  \begin{bmatrix}
    B_{11} & B_{12} & \ldots & B_{1,p_b} \\
    B_{21} & B_{22} & \ldots & B_{2,p_b} \\
    \vdots & \vdots &       & \vdots \\
    B_{p_b,1} & B_{p_b,2} & \ldots & B_{p_b,n_b}
  \end{bmatrix}
\end{align*}
where the matrices $A_{ij}$ and $B_{jk}$ are compatible for matrix
multiplication.  Then we we can write the submatrices of $C$ in terms
of the submatrices of $A$ and $B$
\[
  C_{ij} = \sum_k A_{ij} B_{jk}.
\]

\subsection{The lazy man's approach to performance}

An algorithm like matrix multiplication seems simple, but there is a
lot under the hood of a tuned implementation, much of which has to do
with the organization of memory.  We often get the best ``bang for our
buck'' by taking the time to formulate our algorithms in block terms,
so that we can spend most of our computation inside someone else's
well-tuned matrix multiply routine (or something similar).  There are
several implementations of the Basic Linear Algebra Subroutines
(BLAS), including some implementations provided by hardware vendors
and some automatically generated by tools like ATLAS.  The best BLAS
library varies from platform to platform, but by using a good BLAS
library and writing routines that spend a lot of time in {\em level 3}
BLAS operations (operations that perform $O(n^3)$ computation on
$O(n^2)$ data and can thus potentially get good cache re-use), we can
hope to build linear algebra codes that get good performance across
many platforms.

This is also a good reason to use Julia or \matlab:
they use pretty good BLAS libraries,
and so you can often get surprisingly good performance from it for the types
of linear algebraic computations we will pursue.

\bibliography{refs}
\bibliographystyle{plain}

\end{document}
